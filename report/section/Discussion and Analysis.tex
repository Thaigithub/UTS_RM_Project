\section{Discussion and Analysis}
\subsection{Risk measure interpretion}
The Value at Risk (VaR) and Expected Shortfall (ES) calculated in the previous section represent the potential portfolio losses over specific time horizons (1-day and 10-day) at different confidence levels (90\%, 95\%, and 99\%). VaR estimates the maximum potential loss that is not expected to be exceeded with a certain confidence, while ES provides the average loss given that the VaR threshold has been exceeded. The ES, as a tail risk measure, is generally higher than the VaR, reflecting more conservative estimates of extreme losses.
\subsection{Methods comparison}
\subsubsection{Method 1}
From the above tables, we see that this method returns the highest risk measure for the total portfolio and the MQG portfolio compared to the other methods. In this method, we can see that the risk measures for 1 day are less than 10 days, no matter the method of gathering. In the 10-day risk measure, the non-overlapping method has the highest risk measure and the scaling method has the lowest risk measure. This can be explained as the scaling underestimates the correlation of the log-return by performing the sum (which means that the assumption about the iid of daily log-return does not hold). However, the correlation of daily log-return also makes the risk measure for the overlapping method seem to be lower than the non-overlapping one. This method is easy and quick to use. However, it seems to be naive as it under-estimate the tail distribution. Since then, it will be appropriate to use this method in short-term investing.
\subsubsection{Method 2 and 3}
These two methods have lower risk measure compared to the first method. This can be due to the fact that the assumption of the first method makes the distribution of the log-return bigger compared to what can be achieved via historical data. The risk measure for 1-day risk is less than 10-day risk (which is appropriate). However, method 3 seems to have lower risk measurement. This can be explained as due to the method of age-weighting, it weights less for the loss further in the history (which affected the return during the COVID-19 period). These methods require an enormous amount of data so as to form conduct the measurement. However, it helps consider the historical events and value, which might be underestimated from method 1.
\subsubsection{Method 4 and 5}
These two methods seem to have higher risk measure compared with method 2 and 3 but still lower than the first one. Between these two method, it can be seen that except for the overlapping data gathering method, the risk measurement for by using method 5 is higher compare to method 4. This can be reasonably explained as when using the Student't copula, we tend less to underestimate the tail distribution, which can be seen heavily exhibited in the dependence structure of two log-return of the stock. The explanation for the overlapping method is the same what it has been explained from method 1, where this method will results in the correlation of the data series, which affects on the risk measurement procedure when estimating the characteristic of the distribution of the data. Moreover, as same as method 1, these two methods have the assumption on the distribution of the log-return, which make the tail of the loss fatter than what can be achieved from the historical data. Since then, most of risk measure result from method 4 and 5 is higher than method 2 and 3. However, with method 4, the assumption of the distribution is as same as method 1, but it has a higher approximation (delta-gamma approximation). Since then, the risk measure from method 4 seems to be less than method 1. However, the result from method 5 is bigger than method 1 due to the assumption of Student't distribution of log-return. This method helps less under-estimate the tail distribution from the normal distribution, which helps considering more loss situation in the future. As previously stated, method 4 may underestimate the tail distribution, which happened in 2008 when most investors used Gaussian copula. Method 5 is sufficiently better as it consider a fatter tail distribution but requires more calculation so as to be accurate (more assumption of a long day risk measure as sum of t distribution is not Student't distribution).
\subsection{Diversified vs. Undiversified Portfolio}
From the comparison table of risk measure, we can see that the diversified (total) portfolio helps reduce the risk in all method. However, relying solely on diversified risk measures could mask individual asset risks, especially if correlations fluctuate under stress. In contrast, undiversified measures are useful for identifying specific exposures but may overestimate portfolio risk by ignoring potential diversification benefits.
\subsection{Appraisal and Improvement of Risk Modeling Methods}
Each risk modeling method used in this project has strengths and limitations as discussed above. For improvement:
\begin{itemize}
    \item \textbf{Dependence Modeling:} Incorporating a copula-based approach, such as a Student’s t copula, better captures tail dependence, which is crucial for portfolios with assets likely to experience joint extreme events.
    \item \textbf{Dynamic Volatility Modeling:} Applying techniques like GARCH or EWMA can enhance risk measures by capturing changing market volatility, which is particularly useful for the historical and weighted historical simulation methods.
    \item \textbf{Scenario Analysis and Stress Testing:} Complementing quantitative measures with scenario analysis would provide a broader view of risk, especially in non-normal conditions where traditional VaR and ES might fail.
\end{itemize}
\subsection{Risk Measure Validation}
To validate the risk projections, back-testing can be employed, where actual returns are compared to projected VaR breaches over historical periods. For instance, calculating the frequency of VaR exceedances will indicate the accuracy of the chosen confidence level. Additionally, ES validation could involve conditional coverage tests to assess if ES consistently exceeds VaR breaches, ensuring robustness in extreme market conditions.