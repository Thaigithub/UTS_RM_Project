\section{Executive Summary}
This report provides a comprehensive financial risk analysis for a portfolio comprising a butterfly spread on Commonwealth Bank of Australia (CBA) stock and a strangle on Macquarie Group Ltd. (MQG) stock. The primary objective was to evaluate the portfolio's risk exposure by using Value at Risk (VaR) and Expected Shortfall (ES) at various confidence levels. Considering the projected price movements for CBA (moderate) and MQG (volatile), we employed the Black-Scholes-Merton model for mark-to-market valuation of each position.

Five distinct risk assessment methods were utilized to evaluate the portfolio over one-day and ten-day time horizons: an analytical delta-normal approach, historical simulation, weighted historical simulation, and two Monte Carlo simulations (Gaussian and Student's t copula). Each method provided insights into the portfolio’s risk profile by assessing loss thresholds at 90\%, 95\%, and 99\% confidence levels, both for individual positions and the combined portfolio.

Key insights revealed that the Student’s t copula simulation produced the most conservative estimates, indicating significant tail risk for the portfolio. Additionally, the diversified portfolio's risk was consistently lower than undiversified measures, emphasizing the benefits of diversification. However, the analysis highlights the importance of considering individual asset risks, as well as dynamic modeling approaches to better capture changing market conditions and manage extreme risk scenarios effectively.