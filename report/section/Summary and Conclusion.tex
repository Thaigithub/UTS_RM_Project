\section{Summary and Conclusion}
This report has analyzed the financial risk of a portfolio comprising a butterfly spread on Commonwealth Bank of Australia (CBA) stock and a strangle on Macquarie Group Ltd. (MQG) stock. By employing various methodologies to calculate Value at Risk (VaR) and Expected Shortfall (ES), we have gained insights into the potential risk exposure inherent in the investment strategies.
\begin{itemize}
    \item \textbf{Risk Measure Variability:} The analysis revealed significant differences in risk estimates across the various methods employed. The delta-normal approach tended to understate risk due to its reliance on the assumption of normality, while methods incorporating historical simulations provided more responsive estimates to market dynamics. The Student’s t copula method offered the most conservative ES estimates, highlighting the importance of tail risk considerations.
    \item \textbf{Diversified vs. Undiversified Risks:} The results underscored the value of diversification in managing risk. The diversified portfolio risk measures were consistently lower than those for undiversified assessments, indicating that combining assets can mitigate overall risk exposure. However, reliance on diversified measures should be balanced with awareness of individual asset risks to avoid potential pitfalls during extreme market conditions.
    \item \textbf{Dependence Structures:} The study highlighted the significance of dependence modeling between risk factors. Assets that exhibit strong tail dependence may lead to larger-than-expected losses during market downturns, necessitating the incorporation of robust copula techniques to more accurately model these relationships.
    \item \textbf{Dynamic Risk Management Strategies:} The appraisal of risk modeling methods pointed to opportunities for enhancing risk assessment techniques. Employing dynamic volatility models and scenario analysis can provide a more comprehensive view of risk, allowing for timely adjustments to investment strategies based on changing market conditions.
\end{itemize}
The findings from this analysis can be effectively utilized to inform and manage investment risks in the following ways:
\begin{itemize}
    \item \textbf{Risk Awareness and Assessment:} Investors should adopt a multi-faceted approach to risk measurement, considering both VaR and ES across different methodologies. This allows for a clearer understanding of potential losses and informs decision-making.
    \item \textbf{Portfolio Construction:} The importance of diversification should guide portfolio construction. By carefully selecting assets with low correlations, investors can reduce overall portfolio risk while still pursuing desired returns.
    \item \textbf{Dynamic Adjustments:} Continuous monitoring of market conditions and the performance of risk factors is essential. By employing advanced modeling techniques that account for changing volatility and dependence structures, investors can make more informed decisions and quickly adjust their portfolios to respond to emerging risks.
    \item \textbf{Regulatory Compliance and Reporting:} The insights gained from this analysis also serve to enhance compliance with regulatory requirements related to risk management. Providing robust risk assessments will not only meet regulatory standards but also improve stakeholder confidence in investment strategies.
\end{itemize}
In conclusion, the comprehensive analysis of risk presented in this report provides valuable insights for investors seeking to navigate the complexities of financial markets. By understanding and managing risk effectively, investors can enhance their potential for returns while safeguarding their capital against unforeseen market movements.